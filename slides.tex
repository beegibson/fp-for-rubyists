\documentclass{beamer}

\usepackage[utf8]{inputenc}
\renewcommand{\familydefault}{\sfdefault}
\usepackage{amsmath}

\title{Functional Programming for rubyists}
\author{Bianca Gibson}
\institute{Ruby Conf AU 2016}
\date{}

\begin{document}

\frame{\titlepage}

\begin{frame}
  \begin{center}
    \Huge Rampant side effects
  \end{center}
\end{frame}

\begin{frame}
  \begin{center}
    \Huge Shifting sands of mutable state beneath your feet
  \end{center}
\end{frame}

\begin{frame}
  \begin{center}
    \Huge Enter Functional Programming
  \end{center}
\end{frame}

\begin{frame}
  \frametitle{Who am I?}
  \begin{figure}[p]
    \includegraphics[width=0.8\textwidth]{./assets/portrait.jpg}
  \end{figure}
\end{frame}

\begin{frame}
  \frametitle{What should you get out of this?}
  \begin{itemize}
    \item What is the core of FP?
    \item How can it help with problems you've experienced?
    \item It's not that hard and complicated
    \item No neckbeard required
  \end{itemize}
\end{frame}

\begin{frame}
  \frametitle{What will we cover}
  \begin{itemize}
    \item Immutability
    \item Putting off side effects
    \item Referential transparency
    \item Common traps when explaining to others and avoiding them
    \item How to learn more
  \end{itemize}
\end{frame}

\begin{frame}
  \begin{center}
    \Huge Stop things changing underneath my feet
  \end{center}
\end{frame}
% tell gradle story here

\begin{frame}
  \frametitle{Immutability}
  \begin{itemize}
    \item Don't change any values, use only constants
    \item Instead of changing create a new one
    \item Recursion and accumulators over loops and mutation
    \item Confusion over passing by value or reference? Gone.
    \item Use ! to mutate in ruby
  \end{itemize}
\end{frame}

\begin{frame}
  \frametitle{Immutability example -- fibonacci}
  $f(0) = 1$

  $f(1) = 1$

  $f(n) = f(n - 1) +f(n - 2)$
\end{frame}

\begin{frame}
  \frametitle{What are side effects?}
  \begin{itemize}
    \item When your code affects the outside world
    \item Printing
    \item Writing to disk
    \item network calls
  \end{itemize}
\end{frame}

\begin{frame}
  \frametitle{What do they have to do with FP?}
  \begin{itemize}
    \item Separate your logic from your side effects
    \item Push your side effects out in to the edge of your system
    \item In the middle of your system make representations of the side effects
    \item Makes the core easier to test and reason about
    \item No side effects to stub $\rightarrow$ simpler tests
    \item Push hard testing out to the edge
  \end{itemize}
\end{frame}

\begin{frame}
  \frametitle{Referential transparency}
  \begin{itemize}
    \item The combination of immutability and no side effects
    \item When you call a function again you get the same result
    \item Includes no side effects
    \item You can substitute the saved result of the function for the call
    \item Everything a function does is represented by the result
    \item Nothing unexpected!
  \end{itemize}
\end{frame}

\begin{frame}
  \frametitle{Teaching it}
  \begin{itemize}
    \item Don't get too mathematical
    \item Relate to stuff they already know
% such as monad's they've already used
    \item focus on examples
    \item avoid the scary words like monad until they understand it
% use Venetia story
    \item focus on advantages to them
    \item pick your battles
% some people don't want to hear about it
  \end{itemize}
\end{frame}

\begin{frame}
  \frametitle{Learning more}
  \begin{itemize}
    \item Functional Programming Principles in Scala on coursera
    \item Functional programming in Scala by Paul Chiusano and Rúnar Bjarnason
    \item Refactoring Ruby with Monads by Tom Stuart
    \item github.com/cwmyers/FunctionalTraining
    \item github.com/NICTA/course
    \item If you're after concepts, go for concept oriented resources
    \item Try multiple sources before you find one for you
    \item If it seems really hard, probably the resource, not you.
  \end{itemize}
\end{frame}

\end{document}
